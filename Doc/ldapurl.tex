% $Id: ldapurl.tex,v 1.4 2002/02/09 19:34:02 stroeder Exp $

% ==== 1. ====
% The section prologue.  Give the section a title and provide some
% meta-information.  References to the module should use
% \refbimodindex, \refstmodindex, \refexmodindex or \refmodindex, as
% appropriate.

\section{\module{ldapurl} ---
         LDAP URL handling}

\declaremodule{standard}{ldapurl}

% Author of the module code;
\moduleauthor{python-ldap developers}{python-ldap-dev@lists.sourceforge.net}
% Author of the documentation,
\sectionauthor{Michael Str\"oder}{michael@stroeder.com}

% Leave at least one blank line after this, to simplify ad-hoc tools
% that are sometimes used to massage these files.
\modulesynopsis{Parses and generates LDAP URLs}


% ==== 2. ====
% Give a short overview of what the module does.
% If it is platform specific, mention this.
% Mention other important restrictions or general operating principles.

This module parses and generates LDAP URLs.

\begin{seealso}
\seerfc{2255}{The LDAP URL Format}{}
\end{seealso}
